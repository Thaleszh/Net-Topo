\chapter{Conclusão}
\label{cap:conclusao}

Uma distribuição de carga ciente de topologia pode melhorar o desempenho de uma aplicação por considerar latências de comunicação e efeitos de contenção.
A obtenção de informações de topologia se dá através da coleta de informações dos sistemas em execução e troca de mensagens.
As ferramentas para obtenção destas informações têm limitações e inadequações para o contexto de escalonamento e balanceamento de carga, dificultando seu uso.

Neste trabalho foi desenvolvida a Net Topo, uma abstração de topologia que visa recolher e dispor informações da topologia de rede para uso em escalonamento e balanceamento de aplicações paralelas e distribuídas.
Dentre diversas opções de estruturas de dados e representação de topologia, foi utilizado um CSC, uma matriz de distâncias opcional e uma lista de árvores.
A abstração criada apresenta diversas informações de topologia e fornece funções de proximidade e distâncias, visando auxiliar o processo de distribuição de carga.
A Net Topo também possui diversos mecanismos para sua inicialização, incluindo implementações manuais e importação de um aquivo de topologia em XML.

Foi criado um adaptador para receber informações de inicialização e fazer uma ponte entre a abstração e o \textit{runtime} do \charm, permitindo utilizar a Net Topo junto ao módulo de balanceamento de carga presente na plataforma.
Para comprovar a integração e avaliar o sobrecusto da abstração, foi comparado um balanceador de carga com sua versão modificado para utilizar a Net Topo.
Apesar das execuções serem realizadas em baixa escala e em um ambiente pequeno, elas permitiram observar um sobrecusto de inicialização altamente compensado pelo arquivamento da topologia e pouca ou nenhuma mudança negativa nos tempos de execução do balanceador.
Foram realizados testes unitários para comprovar o uso de funcionalidades que não foram abrangidas no balanceador, de maneira que possam ser utilizadas independente da forma de inicialização da estrutura.

Por fim, foi implementado um gerador de topologias, avaliado o impacto da memoização e observado a escalabilidade da abstração.
O dispositivo de memoização através de uma matriz completa não se mostrou escalável, gerando sobrecustos muito altos conforme o tamanho da topologia cresce.
A função de proximidade se apresentou como constante e o restante das funções testadas demonstraram crescimento linear ou sublinear com o aumento no número de máquinas.

% que problema esse trabalho trata
% que solução foi apresentada
% apresentar os resultados 
% passar pelos objetivos 

\section{Trabalhos Futuros}
\label{sec:futuro}

A execução deste trabalho criou uma série de oportunidades de trabalhos futuros. Segue uma lista de extensões, melhorias e avaliações cogitadas:
\begin{itemize}
    \item Realizar uma distribuição das estruturas de topologia de modo que não seja puramente replicada em cada nó e reduza o espaço utilizado em cada máquina, mas ainda mantenha a possibilidade de uma travessia de grafo para cálculo de \hops ou latência.
    \item Modificar as estruturas da abstração para serem dinâmicas, possibilitando modificações e atualizações em uma topologia prévia, sem ter de reinicializá-la.
    \item Implementar uma inicialização utilizando informações do \hwloc para a topologia de máquina e o \charm para a topologia de rede, retirando as imprecisões relacionadas a máquinas do \charm enquanto se mantém a automatização do processo.
    \item Ampliar o gerador de topologias para a abstração, oferecendo casos de teste para uma gama maior de topologias.
    \item Executar testes de sobrecusto com balanceadores em escalas maiores, avaliando o sobrecusto de outras funcionalidades da Net Topo e como essa se comporta em outros ambientes.
    \item Buscar uma inicialização que contenha informações de latência entre pontos da rede, possivelmente utilizando \netloc.
    \item Aplicar um mecanismo de normalização para relativizar as distâncias na topologia, as deixando mais representativas para mecanismos de escalonamento e balanceamento.
    \item Produzir um mecanismo de agrupamento de regiões da rede, similar ao que foi feito para agrupamento de nós e máquinas, permitindo uma redução nos cálculos de distância e outro nível de proximidade, útil para redes no estilo \Dgfly que possuem partes da rede com latências distintas.
    \item Investigar alternativas ao uso de uma memoização como matriz e avaliar seus benefícios. Como por exemplo um mecanismo de cache um ou um uso de caminhos pré calculados.
\end{itemize}
% distribuir estrutura de maneira não replicada
% Dinamico
% Inicializar com Hwloc + charm para melhores informações
% Opção de retirar memoização
% Geradores de topologia para ampliar os casos de teste e uso
% agrupamento de rede

% investigar uso de uma memoização parcial em vez de completa


