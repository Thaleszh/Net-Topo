\chapter{Introdução}

Este capítulo apresenta o tema de pesquisa do Trabalho de Conclusão de Curso, o escopo no qual o problema em questão será tratado e a justificativa do projeto. 
Na Seção~\ref{sec:introducao} é apresentado o contexto e a motivação para a realização do trabalho. 
Na Seção~\ref{sec:objetivos} são introduzidos os objetivos gerais e específicos, as restrições e premissas existentes, a lista de marcos e os critérios de aceite do projeto. 
Na Seção~\ref{sec:metodologia}, são abordados os métodos de pesquisa envolvidos para alcançar a solução proposta.
Uma visão geral do restante do trabalho é indicada na Seção~\ref{sec:organização}.


\section{Motivação}
\label{sec:introducao}

Aplicações no âmbito científico e industrial incluem cada vez mais detalhes, demandam precisão ainda maior e/ou possuem uma complexidade muito elevada, criando uma demanda cada vez maior por poder de processamento~\cite{pilla-thesis}. Uma das maneiras utilizadas para suprir esta demanda é com computação paralela, que soluciona estes problemas através da repartição de trabalho entre unidades de processamento (\textit{Processing Elements} ou PEs) e a execução destas parcelas simultaneamente. Este método permite que um aumento no número de PEs leve a uma redução no tempo da aplicação. Contudo a melhoria no tempo é limitada pela parcela não paralelizável do programa e pelos recursos do sistema~\cite{amdahl}. 

% Supercomputadores que utilizam da computação paralela foram criados para alcançar processamento massivo e por décadas estas máquinas cresceram. Elas alcançaram um ponto onde se percebeu que velocidade e poder de processamento não eram as métricas adequadas para desempenho destas máquinas. Estes computadores consomem uma quantidade tremenda de energia que é em maior parte dissipada, se transformando em grandes geradores de calor~\cite{green500}. A geração de calor excessiva é um problema em sistema pois temperaturas altas aumentam sua chance de falha e tempo fora de operação~\cite{efficient-metric}. Como contramedida é necessário a implantação de um sistema eficiente de redução de temperatura, que traz outro grande sobrecusto de energia. Com estes problemas em mente a métrica pensada para estas máquinas é a eficiência de processamento, que leva em conta a energia gasta no sistema, a velocidade e o poder de processamento~\cite{efficient-metric}.

Uma das opções para obter eficiência de processamento é o uso de um número maior de núcleos menos potentes~\cite{snir-encyclopedia}. A organização de um número grande de PEs acaba levando a um sistema com memória distribuída, que operam com o uso de diversas máquinas e utilizam alguma rede para a interligação e comunicação. A escolha destes sistemas é devida ao seu custo baixo e a sua grande escalabilidade através do aumento no número de máquinas. O contraponto destas vantagens é o desenvolvimento destas aplicações, que se torna mais complexo pois tem de levar em consideração fatores impactantes como sincronização, dependência e distribuição de dados, balanceamento de carga e custos de comunicação~\cite{pilla-thesis}.

A comunicação em um sistema é realizado através de \links (ligações) entre PEs. Estas ligações estão dispostas utilizando algum padrão de topologia, como malhas, tori, \fatts ou \dgfly, que buscam otimizar alguma característica da rede, como custo de implantação, tráfego ou latência média. 
A alocação de tarefas nestes sistemas geralmente ocasiona um uso compartilhado de \links por múltiplos núcleos, que pode saturar partes da rede devido a uma carga maior de informação a ser transmitida do que os \links suportam, resultando em contenção~\cite{bhatele-encyclopedia}. 
Outro fator de topologia que impacta o desempenho da aplicação é a possibilidade da rede ter ligações com custos heterogêneos~\cite{dragonfly}, tendo latências diferentes para \links diferentes.

%old:  Os PEs em sistemas distribuídos se comunicam usando diversas topologias de rede, como malhas, tori, \fatts ou \dgfly. Os núcleos utilizam de \textit{links} de conexão da rede para efetuar comunicação no sistema. O compartilhamento destes pode resultar em congestão, um congestionamento da rede devido ao sobreuso de alguns \links~\cite{bhatele-encyclopedia}. Outro fator é a possibilidade da rede não ter ligações com custos iguais de comunicação~\cite{dragonfly}, tendo variações de latência e velocidade em partes distintas da rede. O impacto disto é que os caminhos menos custosos entre PEs na rede nem sempre são os menores em questão de \links percorridos.

A repartição de trabalho em ambientes paralelos é um problema a ser tratado, pois aplicações como simulações sísmicas~\cite{dupros,tesser}, dinâmica molecular~\cite{bhatele-kale} ou previsão de tempo~\cite{rodrigues} possuem comportamento dinâmico, ou seja suas cargas mudam ao longo da execução da aplicação, levando um mapeamento homogêneo de tarefas a um estado desbalanceado do sistema, resultando em uma diferença de carga entre alguns PEs. 
Esta diferença faz com que alguns PEs não tenham trabalho para executar enquanto esperam a conclusão de tarefas nos PEs mais carregados, reduzindo o desempenho geral da aplicação. 
Para consertar tais desvios pode-se usar um balanceador de carga (\textit{Load Balancer} ou LB) dinâmico, cujo trabalho é remapear as tarefas entre os PEs, buscando um estado mais balanceado do sistema. 
Este rearranjo garante a utilização dos PEs de maneira mais homogênea e leva a uma redução do tempo da aplicação pela redução de tempo ocioso.

Os caminhos utilizados dentro de uma rede impactam o tempo de execução de um programa, pois caminhos mais lentos resultam em uma latência maior de comunicação e problemas como contenção. É possível que aplicações e LBs levem em conta os custos de comunicação entre os PEs e os custos de movimentação destas tarefas dentro da rede. Tal abordagem pode levar a uma redução do custo de comunicação, evitando a troca de informação e tarefas através de partes lentas da rede. Uma abstração da topologia de rede que permita fácil acesso às informações da rede e de comunicação reduziria a complexidade de criação de aplicações que lidam com contenção e latência.
% olhar netloc para referencia

\textit{Hardware Locality} (\hwloc) e \textit{Network Locality} (\netloc) são duas ferramentas que fornecem uma abstração da topologia para aplicações.
Ambas coletam diversas informações de um sistema, realizam uma interpretação intermediária e a disponibilizam para um usuário, de maneira visual ou em um arquivo \textit{Extensible Markup Language} (XML).
O \hwloc apanha somente a parte de topologia de máquina, enquanto o \netloc coleta informações da rede.
Nenhuma das duas visa auxiliar alocação de tarefas com funções para cálculo de proximidade ou distâncias.
Este trabalho desenvolveu a Net Topo, uma abstração de topologia com foco em auxiliar na alocação de tarefas, detalhado adiante.

\section{Objetivos}
\label{sec:objetivos}

Os objetivos deste trabalho são de desenvolver e avaliar uma abstração da topologia de rede, permitindo seu uso para balanceadores de carga e aplicações distribuídas. Para este fim, seguem os objetivos específicos:

\begin{enumerate}
\item Desenvolver uma API de acesso à informações de topologia de rede;
\item Implementar algoritmos e estruturas para o uso destas informações;
\item Desenvolver uma interface desta abstração utilizando a plataforma Charm++~\cite{website:CHARM};
\item Modificar um LB para que utilize esta base, visando a prova funcional da abstração.

\end{enumerate}

% \begin{flushleft}
% \textbf{Premissas:}
% \begin{itemize}
% \item Um computador estará disponível para realização do trabalho; 
% \item O orientador terá disponibilidade para reuniões periódicas;
% \item Disponibilidade de energia e internet.
% \item Acesso a uma máquina com uma topologia diferenciada para a realização dos testes.
% \end{itemize}

% \textbf{Marcos:}
% \begin{itemize}
 
% \item Entrega do resumo em TCC I: 4ª semana de Novembro/2018; 
% \item Entrega da abstração implementada: 4ª semana de Janeiro/2018;
% \item Entrega da interface com \charm: 1ª semana de Abril/2019;
% \item Entrega do Balanceador modificado: 4ª semana de Abril/2019;
% \item Entrega da primeira versão da monografia em TCC II: inicio da 3ª semana de Junho/2019; 
% \item Defesa da monografia: 1ª semana de Julho/2019;
% \item Entrega da versão final da monografia em TCC II: 4ª semana de Julho/2019.
% \end{itemize}

% \textbf{Critérios de aceite:}
% \begin{itemize}
% \item Aprovação da banca avaliadora;
% \item Aprovação do orientador;
% \item Aprovação do coorientador;
% \item Conformidade da monografia com as normas definidas pela instituição;
% \item Prazos cumpridos.
% \end{itemize}
% \end{flushleft}

\section{Método de Pesquisa}
\label{sec:metodologia}

 
Inicialmente, este trabalho de conclusão foi dividido em duas partes principais: uma de cunho teórico, onde foram estudados os conceitos básicos e o estado da arte para o projeto, e outra mais prática, que envolveu o aprendizado do \textit{framework} \charm. 
A primeira parcela foi constituída por um estudo sobre topologia de rede e balanceamento de carga em aplicações paralelas, criando uma base de conhecimentos para o projeto. 
A segunda parte teve foco no entendimento da plataforma e suas nuances, visando facilitar a implementação da abstração.

A criação da abstração da topologia de rede utiliza a linguagem de programação C++ e é posteriormente inicializada manualmente ou com informações do \fw de programação paralela \charm~\cite{website:CHARM}. 
Esta plataforma possui uma base sólida para a criação, utilização e teste de balanceadores de carga e outras aplicações paralelas~\cite{bhatele-thesis}. 
Além disto, \charm contém somente abstrações de topologia fixas (detalhado na seção~\ref{sec:charm}), carecendo de uma abstração elaborada da topologia de rede, proposta neste trabalho.

Para a criação da abstração de topologia de rede, foi criada uma especificação de suas funções e suas estruturas, criando um esqueleto que foi preenchido posteriormente. 
Um estudo de estruturas e algoritmos utilizados no estado da arte foi realizado para observar como tais funções podem ser implementadas de maneira eficiente.

Após a implementação dos algoritmos e estruturas da abstração de topologia de rede, foi implementado um método de inicialização da estrutura através do \fw \xspace \charm, que coleta as informações necessárias para a execução de seu \textit{runtime}.
Além da inicialização utilizando o \charm, foi implementado uma inicialização utilizando XML e uma inicialização manual.
Para cobrir o funcionamento das funções da abstração independente da forma de inicialização, foram realizados testes de unidade utilizando \textit{Gtest}.

O balanceador de carga \textit{NeighborLB} foi modificado para utilizar esta abstração de rede, demonstrando seu funcionamento e sua utilidade através das funções implementadas e utilizadas.
O balanceador foi comparado com o original, verificando o sobrecusto da abstração criada.

Na última etapa deste trabalho foram realizados testes de desempenho dos algoritmos de inicialização, serialização, proximidade, \hops e distância, avaliando o seu desempenho relativo a entradas diferentes.

\section{Organização do Trabalho}
\label{sec:organização}

O restante do trabalho é dividido em quatro capítulos.
O Capítulo~\ref{cap:conceituacao} apresenta alguns conceitos de importância para o trabalho e duas abstrações de topologia existentes.
No Capítulo~\ref{cap:proposta} a abstração Net Topo é apresentada, incluindo detalhes da estrutura, das funcionalidades e da implementação.
O Capítulo~\ref{cap:comprovacao} descreve um balanceador de carga modificado e os testes realizados para comprovar o funcionamento da abstração e avaliar sua escalabilidade.
O último capítulo apresenta uma conclusão e possíveis trabalhos futuros.
O código implementdo pelo trabalho pode ser encontrado em \href{https://github.com/Thaleszh/Net-Topo}{https://github.com/Thaleszh/Net-Topo}.
% No final do trabalho estão os apêndices com os códigos realizados para este de trabalho.
% O Apêndice~\ref{apd:net_topo} contêm os códigos da abstração e das estruturas que ela utiliza.
% No Apêndice~\ref{apd:testes} estão os testes de unidade e no Apêndice~\ref{apd:lb} os codigos do balanceador modificado, myNeighborLB.
